\chapter{\Large{Introduction}} \label{code}
\subsection{Introduction}

Active circuit contains energy source, such as transistors, diodes, as well as resistors, capacitors and inductors. Many new applications that require high speed and low power consumption deploy the active circuit, embedded in it and make the circuit very complex, which enable us to analyze the circuit. In an active network the main issue is the Identification and stability of the complete circuit. Active network, under certain circumstances, may become unstable and therefore break into oscillation. An improperly designed active circuit can start oscillating or may grow in amplitude with a slight variation of its passive components or active device parameters.This is due to the poles of its transfer function, goes in the right half or Imaginary axis of s-plane.

This dissertation deals with Compensator Design for one port network and certain topics from the area of System Identification. In first section we will use the control theory concepts to design compensating networks for Active network, which are operating in closed loop so as to obtain the desired performance of the overall network and ensuring that the network is stable. This is possible with the help of interconnected networks and application of co-prime factorization theory to design the compensating network to achieve poles of overall network in a left half of s plane and system become stable.

In second Section the Algorithm to formulate the state space equation associated a networks is derived. The solution of this problem, coupled with an ability to solve the state-space equations, means that we can analyze the network using state space equations. In addition state-space formulation for modern electrical circuit analysis is highly important for as it allows the optimal design of linear MIMO systems.

In third section deals with certain topics from the area of System Identification. This area may be defined as a branch of system theory which deals with techniques about how to construct adequate state space models on the basis of available input-output data from a linear time invariant system. After rapid development during thirty years, System Identification is now recognized as an important engineering tool. The theory is embodied in unifying textbooks such as by Ljung \cite{Ljung} and Soderstrom and Stoica \cite{Soder}.

\subsection{Survey of literature}
The notion of open circuit and short stability for one port network defined in \cite{Haykin} and \cite{Chua}. In \cite{Chua}, the definitions of open circuit and short circuit stability are given along with their tests in case of one port networks.  These definitions are used here while solving the stability problem of one port networks. In \cite{Doyle} \cite{Vidya} and \cite{Deoser}, co-prime factorization theory is explained in detail for a control system operating in closed loop with plant transfer function P(s) and controller C(s). An attempt is made in this work to apply this theory to one port linear active networks to design a compensating network.  The state-space formulation for active circuit in \cite{anderson2006network} \cite{Tomas} and is utilized to formulate the state space of passive network and active network.

The whole concept of compensator design, by deploying the application of co-prime factorization theory \cite{Doyle} \cite{Vidya} and \cite{Deoser} was done for one port network.As the initial step our main target was to stabilize one port network. SCILAB Program was developed, which accept the transfer function of given one port network as the input, which provides the compensating network making the overall one port network stable and analyze the overall behavior of the system using bode plot.Two examples were solved to illustrate the application of these concepts to active networks. 

The overall one port network, if not stable, can be made either open circuit or short circuit stable with the help of compensating network connected appropriately in either parallel or series fashion. With the use of co-prime factorization approach, the set of all such compensator which can stabilize the given one port network was determined. The condition under which the overall one port network is open circuit or short circuit stable was derived.\\

Algorithm to find the state space for a given network was drafted with the concepts given in \cite{anderson2006network}. Four different cases of circuits topology and component used were solved as examples. This method uses the network topology to find the state space model of the circuit.

Finally we uses the frequency response analysis to obtain the state space model. It has been found that realization algorithm based on Markov parameters can be effectively applied to the problem of state space identification\cite{ljung1993some} \cite{mckelvey1996subspace}. The impulse response coefficient of the system, also called Markov parameters, are estimated applying the inverse discrete fourier transform(IDFT) to the frequency responce data\cite{kim1999system}. The disadvantage of this approach is that the Markov parameter sequence thus obtained is distorted by time-aliasing effect \cite{Oppenheim}. Another method proposes a simple yet effective way of curve fitting that frequency response function data and of constructing Markov parameters \cite{chen1993frequency}. Once the Markov parameters are constructed, the Eigensystem Realization Algorithm using Data Correlation (ERA/DC) can be used to obtain a state-space model. One difficulty that arises when applying realization algorithm for system Identification is that the Markov parameters are difficult to measure directly.
