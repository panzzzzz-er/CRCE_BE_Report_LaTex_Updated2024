\documentclass{CRCE-BE-Report}
\usepackage{longtable} %for long tables to be fitted on multiple pages
\usepackage{ragged2e}
%\usepackage{color}
\usepackage{graphicx}
\usepackage{varioref}
\usepackage{epsfig}
\usepackage{times}
\usepackage[section]{placeins} % to keep floats(figures) in the section in which they were issued
\usepackage{amsmath}
\usepackage{amssymb}
\usepackage{url}
\usepackage{multirow}
\usepackage[T1]{fontenc}
\usepackage{ascii}
\usepackage{nomencl}
\usepackage{lscape}
\usepackage{longtable,tabu}
\usepackage{lmodern}
\usepackage{lipsum}
\usepackage[tight,footnotesize]{subfigure}
\usepackage{cite}
\usepackage[acronym]{glossaries} 
\usepackage{fancybox}
\usepackage{algorithmic}
\usepackage[boxed]{algorithm}
\usepackage[compact]{titlesec}

%----------For code listing---------------------------------------------------
\usepackage{listings} % Required for inserting code snippets
\usepackage[usenames,dvipsnames]{color} % Required for specifying custom colors and referring to colors by name

\definecolor{DarkGreen}{rgb}{0.0,0.4,0.0} % Comment color
\definecolor{highlight}{RGB}{255,251,204} % Code highlight color

\lstdefinestyle{Style1}{ % Define a style for your code snippet, multiple definitions can be made if, for example, you wish to insert multiple code snippets using different programming languages into one document
language=Scilab, % Detects keywords, comments, strings, functions, etc for the language specified
backgroundcolor=\color{highlight}, % Set the background color for the snippet - useful for highlighting
basicstyle=\footnotesize\ttfamily, % The default font size and style of the code
breakatwhitespace=false, % If true, only allows line breaks at white space
breaklines=true, % Automatic line breaking (prevents code from protruding outside the box)
captionpos=b, % Sets the caption position: b for bottom; t for top
commentstyle=\usefont{T1}{pcr}{m}{sl}\color{DarkGreen}, % Style of comments within the code - dark green courier font
deletekeywords={}, % If you want to delete any keywords from the current language separate them by commas
%escapeinside={\%}, % This allows you to escape to LaTeX using the character in the bracket
firstnumber=1, % Line nutmbers begin at line 1
frame=single, % Frame around the code box, value can be: none, leftline, topline, bottomline, lines, single, shadowbox
frameround=tttt, % Rounds the corners of the frame for the top left, top right, bottom left and bottom right positions
keywordstyle=\color{Blue}\bf, % Functions are bold and blue
morekeywords={}, % Add any functions no included by default here separated by commas
numbers=left, % Location of line numbers, can take the values of: none, left, right
numbersep=10pt, % Distance of line numbers from the code box
numberstyle=\tiny\color{Gray}, % Style used for line numbers
rulecolor=\color{black}, % Frame border color
showstringspaces=false, % Don't put marks in string spaces
showtabs=false, % Display tabs in the code as lines
stepnumber=5, % The step distance between line numbers, i.e. how often will lines be numbered
stringstyle=\color{Purple}, % Strings are purple
tabsize=2, % Number of spaces per tab in the code
}

% Create a command to cleanly insert a snippet with the style above anywhere in the document
\newcommand{\insertcode}[2]{\begin{itemize}\item[]\lstinputlisting[caption=#2,label=#1,style=Style1]{#1}\end{itemize}} % The first argument is the script location/filename and the second is a caption for the listing

%--------------------------------------End of code listing-------------------

\titlespacing{\section}{0pt}{2ex}{1ex}
\titlespacing{\subsection}{0pt}{1ex}{0ex}
\titlespacing{\subsubsection}{0pt}{0.5ex}{0ex}

\usepackage[compact]{titlesec}
\titleformat{\chapter}[display]
  {\normalfont\huge\bfseries\centering}{\chaptertitlename\ \thechapter}{20pt}{\LARGE}

%\usepackage[toc,page]{appendix}

\makeglossaries
\makenomenclature

\newglossaryentry{wsn}{name=WSN,description=Wireless Sensor Network}
\newglossaryentry{manet}{name=MANET,description=Mobile Ad hoc NETwork}
\newglossaryentry{gps}{name=GPS,description=Global Positioning System}
\newglossaryentry{spin}{name=SPIN,description=Sensor Protocols for Information via Negotiation}
\newglossaryentry{dc}{name=DC,description=Data Centric}
\newglossaryentry{bs}{name=BS,description=Base Station}
\newglossaryentry{mcfa}{name=MFCA,description=Minimum Cost Forwarding Algorithm}
\newglossaryentry{acquire}{name=ACQUIRE,description=Active Qwery Forwarding in Sensor Networks}
\newglossaryentry{leach}{name=LEACH,description=Low Energy Adaptive Clustering Hierarchical}
\newglossaryentry{leachc}{name=LEACH-C,description=LEACH Centralized}
\newglossaryentry{pegasis}{name=PEGASIS,description=Power Efficient Gathering in Sensor Information Systems}
\newglossaryentry{teen}{name=TEEN,description=Threshold-Sensitive Energy Efficient SensormNetwork Protocol}
\newglossaryentry{heed}{name=HEED,description=Hybrid Energy-Efficient Distributed}
\newglossaryentry{gaf}{name=GAF,description=Geographic Adaptive Fidelity}
\newglossaryentry{gear}{name=GEAR,description=Geographic and Energy Aware Routing}
\newglossaryentry{sleachc}{name=sLEACH-C,description=Solar-aware LEACH-centralized extension}
\newglossaryentry{group}{name=GROUP,description=Genetic algorithm inspired ROUting Protocol}
\newglossaryentry{aco}{name=ACO,description=Genetic algorithm inspired ROUting Protocol}
\newglossaryentry{pso}{name=PSO,description=Particle swarm optimization}

\newglossaryentry{aodv}{name=AODV,description=Ad hoc On-demand Distance Vector}
\newglossaryentry{dsr}{name=DSR,description=Dynamic Source Routing}
\newglossaryentry{rfc}{name=RFC,description=Request For Comments}
\newglossaryentry{ara}{name=ARA,description=Ant-Inspired Routing Algorithm}
\newglossaryentry{ec}{name=EC,description=Evolutionary Computation}
\newglossaryentry{ga}{name=GA,description=Genetic Algorithms}
\newglossaryentry{ils}{name=ILS,description=Iterated Local Search}
\newglossaryentry{sa}{name=SA,description=Simulated Annealing}
\newglossaryentry{ts}{name=TS,description=Tabu Search}
\newglossaryentry{si}{name=SI,description=Swarm Intelligence}
\newglossaryentry{as}{name=AS,description=Ant System}
\newglossaryentry{eas}{name=EAS,description=Elitist strategy for Ant System}
\newglossaryentry{tsp}{name=TSP,description=Traveling Salesman Problem}
\newglossaryentry{so}{name=SO,description=Self-Organization}
%%%%%%%%%%%%%%%%%%%%%%%%%%%%%MAIN PART OF THE REPORT%%%%%%%%%%%%%%%%%%%%%%%%%%%%%%%
\begin {document}
\graphicspath{{./}{images/}}
\headheight-8pt % to adjust white space.
\headsep -12pt %
\footskip 18mm %
% prelude.tex
%   - titlepage
%   - dedication (optional)
%   - approval sheet
%   - table of contents, list of tables and list of figures
%   - abstract
%============================================================================


\clearpage\pagenumbering{roman}  % This makes the page numbers Roman (i, ii, etc)


% TITLE PAGE
%   - define \title{} \author{} \date{}
\title{Harmony Guard: Foul Language Detection for Hindi Audio}
%\authora{Shalini Shrivastava}
\stua{Pratham Mahajan}
\stub{Mohit Pansare}
\stuc{Sakshee Patil}


\date{\large April 24, 2024} %\today

%  - Roll number, required for title page, approval sheet, and
%    certificate of course work 

\rollnuma{9383}
\rollnumb{9391}
\rollnumc{9393}
%\rollnumd{2222}
%\rollnumb{1023}

%   - The default degree is ``Doctor of Philosophy''
%     (unless the document style msthesis is specified
%      and then the default degree is ``Batchlor of Engineering'')
%     Degree can be changed using the command \mudegree{}
\mudegree{Bachelor of Engineering}

%   - The default report type is preliminary report.
%      * for a PhD thesis, specify \thesis
%\thesis
%      * for a M.Tech./M.Phil./M.Des./M.S. dissertation, specify \dissertation
\project
%      * for a DIIT/B.Tech./M.Sc.project report, specify \project
%\project
%      * for any other type, use  \reporttype{}
%\reporttype{ReportType}

%   - The default department is ``Unknown Department''
%     The department can be changed using the command \department{}
\department{DEPARTMENT OF ARTIFICIAL INTELLIGENCE AND DATA SCIENCE}

%\section{Graphics}
%    - Set the guide's name
\setguide{Prof. Swapnali Makdey}
%    - Set the coguide's name (if you have one)
%\setcoguide{PPP}
%    - Set external guide (if you have one)
%\setexguide{Prof External Guide}

%   - once the above are defined, use \maketitle to generate the titlepage
\maketitle

%--------------------------------------------------------------------%
% DEDICATION
%   Dedications, if any, must be first page after title page.
\begin{dedication}
 \bf \it This work is dedicated to my family.\\I am very thankful for their motivation and support.
%I appreciate and am very thankful for their continued motivation and support. 
\end{dedication}

%--------------------------------------------------------------------%
% APPROVAL SHEET
%   - for final thesis, you need Approval Sheet. So, uncomment the
%     \makeapproval command.
%     it should come after dedication, if dedication is
%     present. Otherwise it is the first page after title page.

\makecertificate

\makeapproval



%--------------------------------------------------------------------%
% COPYRIGHT PAGE
%   - To include a copyright page use \copyrightpage
% \copyrightpage

%--------------------------------------------------------------------%
% ABSTRACT

\thispagestyle{empty}
%\input{declaration.tex}
    \makedeclaration

\justify{

\abstract

}


%--------------------------------------------------------------------%
% CONTENTS, TABLES, FIGURES
%\tableofcontents
%\listoftables
%\listoffigures

%--------------------------------------------------------------------%
% NOMENCLATURE
%\begin{nomenclature}
%\begin{description}
%\item{\makebox[0.75in][l]{$C_1$}} Constant 1
%
%\item{\makebox[0.75in][l]{$V$}}    Voltage 
%
%\item{\makebox[0.75in][l]{\$}}     US Dollars
%\end{description}
%\end{nomenclature}
%
%\cleardoublepage\pagenumbering{arabic} % Make the page numbers Arabic (1, 2, etc)
   %%Make changes in this file for title of the project, abstract and details of the author
\begin{acknowledgments}

\thispagestyle{empty}

\noindent We have great pleasure in presenting the report on {\bf "\projecttitle"}. I take this opportunity to express my sincere thanks towards the guide Prof. Swapnali Makdey, Fr. C.R.C.E, Bandra (W), Mumbai, for providing the technical guidelines, and the suggestions regarding the line of this work. We enjoyed discussing the work progress with him during our visits to department.\\

\noindent We thank Dr. Jagruti Save, Head of Artificial Intelligence and Data Science Dept., Principal and the management of Fr. C.R.C.E., Mumbai for encouragement and providing necessary infrastructure for pursuing the project.\\

\noindent We also thank all non-teaching staff for their valuable support, to complete our project.
\par
%

%\hfill Vaibhav Godbole

\end{acknowledgments}
%I thank the many people who have done lots of nice things for me.



\newpage
\tableofcontents
\listoffigures
\listoftables
\printglossaries
\addcontentsline{toc}{chapter}{Glossary}
%\addcontentsline{toc}{chapter}{List of Abbreviations}
\printnomenclature[2in]
\cleardoublepage\pagenumbering{arabic}
\pagestyle{plain}
\chapter{\Large{Introduction}} \label{code}
\subsection{Introduction}

Active circuit contains energy source, such as transistors, diodes, as well as resistors, capacitors and inductors. Many new applications that require high speed and low power consumption deploy the active circuit, embedded in it and make the circuit very complex, which enable us to analyze the circuit. In an active network the main issue is the Identification and stability of the complete circuit. Active network, under certain circumstances, may become unstable and therefore break into oscillation. An improperly designed active circuit can start oscillating or may grow in amplitude with a slight variation of its passive components or active device parameters.This is due to the poles of its transfer function, goes in the right half or Imaginary axis of s-plane.

This dissertation deals with Compensator Design for one port network and certain topics from the area of System Identification. In first section we will use the control theory concepts to design compensating networks for Active network, which are operating in closed loop so as to obtain the desired performance of the overall network and ensuring that the network is stable. This is possible with the help of interconnected networks and application of co-prime factorization theory to design the compensating network to achieve poles of overall network in a left half of s plane and system become stable.

In second Section the Algorithm to formulate the state space equation associated a networks is derived. The solution of this problem, coupled with an ability to solve the state-space equations, means that we can analyze the network using state space equations. In addition state-space formulation for modern electrical circuit analysis is highly important for as it allows the optimal design of linear MIMO systems.

In third section deals with certain topics from the area of System Identification. This area may be defined as a branch of system theory which deals with techniques about how to construct adequate state space models on the basis of available input-output data from a linear time invariant system. After rapid development during thirty years, System Identification is now recognized as an important engineering tool. The theory is embodied in unifying textbooks such as by Ljung \cite{Ljung} and Soderstrom and Stoica \cite{Soder}.

\subsection{Survey of literature}
The notion of open circuit and short stability for one port network defined in \cite{Haykin} and \cite{Chua}. In \cite{Chua}, the definitions of open circuit and short circuit stability are given along with their tests in case of one port networks.  These definitions are used here while solving the stability problem of one port networks. In \cite{Doyle} \cite{Vidya} and \cite{Deoser}, co-prime factorization theory is explained in detail for a control system operating in closed loop with plant transfer function P(s) and controller C(s). An attempt is made in this work to apply this theory to one port linear active networks to design a compensating network.  The state-space formulation for active circuit in \cite{anderson2006network} \cite{Tomas} and is utilized to formulate the state space of passive network and active network.

The whole concept of compensator design, by deploying the application of co-prime factorization theory \cite{Doyle} \cite{Vidya} and \cite{Deoser} was done for one port network.As the initial step our main target was to stabilize one port network. SCILAB Program was developed, which accept the transfer function of given one port network as the input, which provides the compensating network making the overall one port network stable and analyze the overall behavior of the system using bode plot.Two examples were solved to illustrate the application of these concepts to active networks. 

The overall one port network, if not stable, can be made either open circuit or short circuit stable with the help of compensating network connected appropriately in either parallel or series fashion. With the use of co-prime factorization approach, the set of all such compensator which can stabilize the given one port network was determined. The condition under which the overall one port network is open circuit or short circuit stable was derived.\\

Algorithm to find the state space for a given network was drafted with the concepts given in \cite{anderson2006network}. Four different cases of circuits topology and component used were solved as examples. This method uses the network topology to find the state space model of the circuit.

Finally we uses the frequency response analysis to obtain the state space model. It has been found that realization algorithm based on Markov parameters can be effectively applied to the problem of state space identification\cite{ljung1993some} \cite{mckelvey1996subspace}. The impulse response coefficient of the system, also called Markov parameters, are estimated applying the inverse discrete fourier transform(IDFT) to the frequency responce data\cite{kim1999system}. The disadvantage of this approach is that the Markov parameter sequence thus obtained is distorted by time-aliasing effect \cite{Oppenheim}. Another method proposes a simple yet effective way of curve fitting that frequency response function data and of constructing Markov parameters \cite{chen1993frequency}. Once the Markov parameters are constructed, the Eigensystem Realization Algorithm using Data Correlation (ERA/DC) can be used to obtain a state-space model. One difficulty that arises when applying realization algorithm for system Identification is that the Markov parameters are difficult to measure directly.

\newcommand{\tab}{\hspace*{2em}}
\chapter{Literature Review} 
\noindent In this section, some related work on MANET routing algorithms, both conventional and ant-inspired, will be discussed

\chapter{Problem Statement}
\section{Drawbacks of Ant Colony Optimization Algorithm}
\chapter{Project Description}
\noindent In this section you should explain your project in detail
\section{Overview of the project}
\noindent In this section you should explain the flow chart of your project
\section {Module Description}
\subsection {Modules}
\noindent In this section you should discuss the modules and sub-modules in detail
\subsection {Data Flow Diagram}
\subsection{E-R Diagram}
\subsection{Database Design}
\subsubsection{Table 1}
\subsubsection{Table 2}
\subsection {Input Design}
\noindent Here is input design
\subsection {Output Design}
\noindent Here is Output Design
are sent to these neighboring nodes. Algorithm \ref{test} shows our proposed algorithm \cite{vaibhav1}.\\

\singlespace
{
\begin{algorithm}[h!]

 \caption{Proposed Algorithm - TEST}
 \label{test}

\begin{algorithmic}[1]

\STATE {\it t}; current time
\STATE {\it t$_{end}$}; time length of simulation 
\STATE {\it $\mathcal{4}$t}; time interval between ants generation (ant timer)
        \FOR{$i \in C $}
            \STATE\it $\mathcal{M}$ $\leftarrow$ InitLocalTrafficModel
            \STATE \it $\mathcal{T}$ $\leftarrow$ InitNodeRoutingTable
            \WHILE {t $\le$ t$_{end}$}
            	\IF {\it t mod $\mathcal{4}$ t = 0}
            		\STATE destination $\leftarrow$ SelectDestination (traffic distribution at source)
            		\STATE LaunchForwardAnt(source, destination)
            \ENDIF
            	\FOR {ActiveForwardAnt[source, current, destination]}
            	\WHILE {current $\ne$ destination}
            	\IF {\it r $<$ set value}
            		\STATE find closest node which is nearest to destination using Range Difference GPS technique
            	\ELSE
            	\STATE next hop $\leftarrow$ SelectLink(current, destination, link queues, $\mathcal{T}$)
            	\STATE PutAntOnLinkQueue(current, next hop)
            	\STATE WaitOnDataLinkQueue(current, next hop)
            	\STATE CrossLink(current, next hop)
            	\STATE Memorize(next hop, elapsed time)
            	\STATE current $\leftarrow$ next hop
            	\ENDIF
            	\ENDWHILE
            	\STATE LaunchBackwardAnt(destination, source, memory data)
            	\ENDFOR
            \FOR {ActiveBackwardAnt[source, current, destination]}
            \WHILE {current $\ne$ destination}
            \STATE next hop $\leftarrow$ pop memory
            \STATE WaitOnHighPriorityLinkQueue(current, next hop)
            \STATE CrossLink(current, next hop)
            \STATE from $\leftarrow$ current
            \STATE current $\leftarrow$ next hop
            \STATE 
            \STATE r $\leftarrow$ GetNewPheromone ($\mathcal {M}$, current, from, source, memory data)
            \STATE UpdateLocalRoutingTable($\mathcal {T}$, current, source, r)
            \ENDWHILE
            \ENDFOR
            \ENDWHILE 
          \ENDFOR
 \end{algorithmic}
\end{algorithm}
}
\chapter{System Design}

This is test
\section{Unit Testing}
This is testThis is testThis is testThis is testThis is testThis is testThis is test

\section{Acceptance Testing}
This is testThis is testThis is testThis is test

\section{Test Cases}
This is testThis is testThis is testThis is testThis is test
\chapter{Implementation Details}
\section{Methodology}
In this section explain the implementation methodology in detail
\chapter{Results}
\subsection{Result Analysis}
This is testThis is testThis is testThis is testThis is testThis is testThis is testThis is testThis is testThis is test

\subsection{conclusion}

Write your conclusion in brief. 

\subsection{Future Enhancements}
\par The future vision of WSNs is to embed numerous distributed devices to monitor and interact with physical world phenomena, and to exploit spatially and temporally dense sensing and actuation capabilities of those sensing devices. These nodes coordinate among themselves to create a network that performs higher-level tasks.
\par Further reasearch is needed to modify the existing routing protocols to provide QOS. In genetics, Particle swarm optimization \gls {pso} can be used to develop the algorithm for packet scheduling.






\chapter{Conclusions and Future Enhancements}
%\input{Chapter2.tex}
%\input{Stability.tex}
%\input{StateSpace.tex}
%\input{CircuitExample.tex}
%\input{SolvedEXP.tex}
%\input{FRSS.tex}
%\input{con.tex}
%\renewcommand\bibname{References}
\addcontentsline{toc}{chapter}{Appendix}
\appendix
\chapter{Appendix}
\insertcode{"scilab.sci"}{A sample Scilab code} % The first argument is the script location/filename and the second is a caption for the listing

%---------------------------------------------------------------------------------------
%\insertcode{"scilab.sci"}{A sample Scilab code} % The first argument is the script location/filename and the second is a caption for the listing
%\nocite{*}
\renewcommand\bibname{References}
\bibliographystyle{IEEEtran}
\bibliography{References}
\addcontentsline{toc}{chapter}{References}
%\setcounter{toc}{-1}


\end{document}
